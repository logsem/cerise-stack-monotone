\documentclass{article}
\usepackage[utf8]{inputenc}
\usepackage{hyperref}
\usepackage{amsmath}
\usepackage{amssymb}
\usepackage{mathpartir}
\usepackage{tensor}
\usepackage{xspace}
\usepackage[dvipsnames]{xcolor}
\usepackage{iris}
\usepackage{marvosym}
\usepackage{xargs}

\setlength{\parskip}{0.3em}

\newcommand{\Z}{\mathbb{Z}}
\newcommand{\X}[1]{\ensuremath{\mathrm{#1}}}
\newcommand{\V}[1]{\ensuremath{\mathit{#1}}}
\newcommand{\I}[1]{\ensuremath{\mathtt{#1}}}
\newcommand{\T}[1]{\texttt{#1}}
\newcommand{\SL}{Separation Logic\xspace}
\newcommand{\pure}[1]{\tensor[^{\ulcorner}]{#1{}}{^{\urcorner}}} %Hacky {} to avoid double superscript

\newcommand{\FIXME}[1]{{\color{MidnightBlue} FIXME: #1}}

\newcommand{\MMIO}{\textlog{MMIO}}

\DeclareMathOperator{\initOKo}{init_{OK}}
\newcommandx{\initOK}[4][3=r,4=m]{\initOKo(#1,#2,#3,#4)}
\DeclareMathOperator{\initBCo}{init_{BC}}
\newcommandx{\initBC}[4][3=r,4=m]{\initBCo(#1,#2,#3,#4)}

\newcommand{\regdr}{\mathbin{!!}}


\title{Capabilities, MMIO and Robust Safety}
\date{March 10, 2020}

\begin{document}

\maketitle

\section{Memory Mapped I/O: Operational Semantics}

The proposal is to simply represent read and writes to memory-mapped IO
addresses as events in a trace. This says nothing a priori about devices that
might be connected to these IO regions. In particular, without additional
assumptions, reading a byte from a memory-mapped region just returns an
arbitrary value.

If at some point we want to reason under the assumption that we are connected to
a specific device that reacts in a specific way, we can express that as an extra
\SL assertion, that we assume as a pre-condition, and that restricts the set of
different traces that we might observe.

\[
  \begin{array}{lcl}
    \X{EventTy} & := & \X{IOWrite} \; | \; \X{IORead} \\
    \X{Event} & := & \X{EventTy} \times \X{Addr} \times \Z \\
    \X{Trace} & := & \X{list} \; \X{Event} \\
    \X{State} & := & \underbrace{\X{Reg} \times \X{Mem}}_{\text{old state}}
                     \times \X{Trace} \\
  \end{array}
\]

Values of type \X{State} represent the state of a configuration in the
small-step operational semantics. In this setup, we assume the whole semantics
to be parameterized by the range of memory-mapped addresses: \MMIO.

\[
  \begin{array}{lcl}
    \MMIO & := & [\MMIO_{\X{b}}, \MMIO_{\X{e}}) \\
  \end{array}
\]

\textit{NB:} An alternative presentation would be to make this range an
immutable part of \X{State}. It probably does not really matter in the end, we
should do whatever makes our life easier for instantiating Iris with the
semantics.

In the (current) operational semantics without MMIO, the operational semantics
of the \I{Load} instruction is:

\begin{mathpar}
  \inferrule[Load]
  { r[\X{src}] = (p,g,b,e,a) \\ \X{readAllowed}\; p \\ a \in [b,e) }
  { (r, m) \xrightarrow{\I{Load} \; \X{dst} \; \X{src}} (r[\X{dst} := m[a]], m) }
\end{mathpar}

With MMIO, we obtain two rules:

\begin{mathpar}
  \inferrule[MemLoad]
  { r[\X{src}] = (p,g,b,e,a) \\ \X{readAllowed}\; p \\ a \in [b,e) \\ a \notin \MMIO}
  { (r, m, t) \xrightarrow{\I{Load} \; \X{dst} \; \X{src}}
    (r[\X{dst} := m[a]], m, t) }

  \inferrule[IOLoad]
  { r[\X{src}] = (p,g,b,e,a) \\ \X{readAllowed}\; p \\ a \in [b,e) \\ a \in \MMIO}
  { (r, m, t) \xrightarrow{\I{Load} \; \X{dst} \; \X{src}}
    (r[\X{dst} := x], m, (\X{IORead}, a, x) :: t) }
\end{mathpar}

Notice how in the second rule, we read an arbitrary integer $x$, and record it
in the trace.

The \I{Store} rule would be similar. Additionally, executing with a \X{PC}
register pointing to an $\MMIO$ location should be disallowed in the operational
semantics (I think this makes more sense than reading a random instruction
through I/O).


\paragraph*{Remark:} With this presentation, the specification of the
``machine'' is very much decoupled from the model of the devices it might be
communicating with. This is a good thing, I think.

Nevertheless, one could consider an alternative presentation where the model of
the devices is more tightly integrated with the semantics of the machine. For
instance, one could make the operational semantics parameterized with the
devices' model, where each device is modeled as having some internal state, the
ability to react on reads or writes, or to perform an internal step. Then, the
operational semantics would either step the usual way, or whenever a device
steps.

I believe this would be somewhat similar to the semantics of I/O system calls
through a foreign function interfaces as formalized in
CakeML~\cite{cakeml-vstte17io}.

\paragraph*{Remark:} Alix says that the proposed style of operational semantics
(using a trace) is close to the semantics of \texttt{volatile} as in
CompCert---which is also how memory mapped addresses seem to be exposed to C
compilers in practice. So this is probably a good sign. 


\section{\SL Resources}

We added a trace as part of the state, so we wish to also expose it as a \SL
assertion.

\newcommand{\tracefull}[1]{\ownGhost{\gamma_{\X{T}}}{\authfull{#1}}}
\newcommand{\tracefrag}[1]{\ownGhost{\gamma_{\X{T}}}{\authfrag{#1}}}

The simplest way to do that is to directly expose the complete trace. In that
case, the state interpretation additionally holds a resource for the trace:

\[
  \X{state\_interp} \; (r,m,t) := \ldots \ast \tracefull{t}
\]

The user then works with assertions of the form $\tracefrag{t'}$. Such an
assertion is not duplicable, and grants full ownership over the trace. In
particular, it allows one to \emph{update} the trace by emitting events, i.e. by
performing I/O operations.

The wp-rules for \I{Load} and \I{Store} need to be updated consequently. For
instance, the rule for a \I{Load} reading the integer $x$ on a MMIO address $a$
now requires $\tracefrag{t}$ in the pre-condition (for some $t$), and provides
$\tracefrag{(\X{IORead}, a, x) :: t}$ in the post-condition.

The corresponding resource algebra is $\authm(\exm(\X{Trace}))$. A few relevant
rules are:

\[
  \begin{array}{l}
    \tracefull{t} \ast \tracefrag{t'} \wand \pure{t = t'} \\
    \tracefrag{t} \ast \tracefrag{t'} \wand \FALSE \\
    \tracefull{t} \ast \tracefrag{t} \vsW \tracefull{t'} \ast \tracefrag{t'} \\
  \end{array}
\]


\paragraph*{Remark:} This is very coarse-grained: either one has the full
ownership for performing I/O and reasoning about it, or one cannot know anything
about it.

A first extension could be to allow observing prefixes of the trace (since
events can only be appended to the trace). The observation that the trace
has a given prefix would be duplicable. This again seems to be an application of
monotonicity, that could be realized by having a duplicable AtLeast part as part
of $\tracefrag{t'}$,
similarly to how we currently model Monotone References.

Another extension, that seems very useful for modular reasoning, would be to
allow splitting the trace along separate range of addresses. Concretely, a trace
containing events about the range of MMIO addresses $[a,c)$ could be split into
two traces, granting ownership over events on addresses $[a,b)$ and $[b,c)$
respectively. Note that recombining these two traces would only yield some
unspecified interleaving of the events from the two traces, and not necessarily
yield the original trace, since in our model, we cannot know the exact interleaving of IO operations issued by different parties.

One application of this second extension could be the verification of an example
involving ``multiplexed'' I/O, where two separate parts of the code are granted
separate ownership over separate MMIO addresses. These two separate pieces of
code would be verified separately; then, in the end, one could prove that one
gets \emph{some} interleaving of all the events emitted by both components.

We use another resource algebra to model the MMIO locations. The operational
semantics have been parameterized with a set $\MMIO$ of memory locations, but we
need a way to specify what these locations are at a higher level of abstraction.
A non-elegant solution is to index the WP definition with this set $\MMIO$. A
better solution (that does not necessitate any changes to the definition of WP)
is to add $\MMIO$ to the state, and add a conjunct to the state interpretation
that expresses ownership of $\MMIO$ under agreement, i.e. add

\newcommand{\MMIOag}{\ownGhost{\gamma_{\X{MMIO}}}{\agm (\X{MMIO})}}

\[
  \X{state\_interp} \; (r,m,t,\MMIO) := \ldots \ast \MMIOag
\]

to the state interpretation we had before.

\paragraph*{Remark:} Another alternative, mentioned by Dominique, was to
parameterize the entire language by the set of MMIO locations. This might come
down to the same thing as the first solution mentioned above.

In the future, we might be interested in the dynamic allocation of MMIO memory.
The question is what this would mean, though, since the set of available devices
allowing for MMIO access and the concrete buffers they provide will most likely
still be modeled as fixed at runtime.
Rather, this would allow us to take (un)mapped MMIO locations, connected to the different devices, and (un)map them anywhere in main memory.
This would require a different way of modeling MMIO, where we need both a notion
of the total pool of possible MMIO locations, and a description of the currently
mapped locations. We could model this using an authoritative RA.

\section{Toplevel Theorem (\emph{à la} OCPL)}

Let us start with a brief recap of the toplevel ``Robust Safety'' theorem for
OCPL itself (and its key ingredients), then the extension (by Thomas) of OCPL to
include a \I{print} capability, and finally move to the capability machine
setting.

\subsection{OCPL}

The {\sc RobustSafety} theorem of OCPL is as follows:

\begin{mathpar}
  \mprset{vskip=0.3em}
  \inferrule
  {C \in \textit{AdvCtx} \\
    e \; \X{closed} \\
    \hoare{\TRUE}{e}{x.\; \textlog{lowval}\; x} \\
    (C[e]);(\emptyset, \textlog{OK}) \longrightarrow^* T';(h',g')
  }
  {g' = \textlog{OK}}
\end{mathpar}

For any closed expression $e$, if $e$ has been verified to only return low
values, then running $e$ wrapped in an adversarial context $C$ from an initial
state, then we can observe that every reachable state is good
($g' = \textlog{OK}$ means that no assertion has failed in $e$).

$C \in AdvCtx$ means that $C$ cannot contain assertions (otherwise one could
trivially contradict the theorem by taking
$C[\cdot] = \textlog{assert false}$), and cannot contain references to raw
memory locations (otherwise one could directly access $e$'s private state,
invalidating the local state encapsulation mechanisms).

\newcommand{\lift}[2]{\textlog{lift}\; #1 \; #2}
\newcommand{\liftP}[1]{\lift{\Psi}{#1}}

$\textlog{lowval}\; x$ intuitively means that $x$ cannot be used to directly
access private (or ``high'') locations. It is formally defined using a logical
relation ``$\liftP{v}$'', which more generally asserts that the value $v$ only
gives direct access to locations that satisfy the predicate $\Psi$.

\[
  \begin{array}{lcl}
    \liftP (\textlog{rec}\; f\; x.\; e) & \eqdef
    & \later \forall v.\; \hoare{\liftP{v}}{e[v/x,\textlog{rec}\;f\;x.\; e/f]}{y.\; \liftP y} \\
    %
    \liftP (v_{1}, v_{2}) & \eqdef & \later (\liftP v_{1}, \liftP v_{2}) \\
    %
    \color{BrickRed} \liftP \ell & \color{BrickRed} \eqdef & \color{BrickRed} \Psi \; \ell \\
    %
    \ldots
  \end{array}
\]

Then, $\textlog{lowval}\; x$ is defined as $\lift{\textlog{lowloc}}{x}$, where
$\textlog{lowloc}$ is a predicate characterizing the region of memory containing
``low locations'' (distinct from the other region containing ``high
locations'').

\subsection{OCPL with \I{print}}

\newcommand{\OutV}{\textlog{Out}}

In Thomas' extension of OCPL, a new value $\OutV$ is added, denoting an
``output object capability'', as well as a $\textlog{print}$ primitive, where
$\textlog{print}\; \OutV\; v$ effectively ``prints'' the value $v$, i.e. adds it
to the trace of printed values.

One then wants to be able to encapsulate the use of $\OutV$, for instance by
defining object capabilities that enforce some invariants on the values being
printed. In that setting, $\OutV$ is considered as a ``high'' value:

\[
  \begin{array}{lcl}
    \liftP \OutV & \eqdef & \FALSE
  \end{array}
\]

To be more precise, we could actually allow sharing $\OutV$ in case the predicate P
does not place any constraints on the output, i.e. is the $\TRUE$ predicate.
Technically, we could hence have the following definition (although admittedly,
it does look a bit hard-coded):
\[
  \begin{array}{lcl}
    \liftP \OutV & \eqdef & \exists \iota \ldotp \knowInv{\iota}{\exists t\ldotp \tracefrag{t} \ast
                            \pure{(\Lam{\_}.\TRUE) \spac t}}
  \end{array}
\]

The theorem then becomes:

\begin{mathpar}
  \mprset{vskip=0.4em}
  \inferrule
  {C \in \textit{AdvCtx} \\
    e \; \X{closed} \\
    \knowInv{\iota}{\exists t.\; \tracefrag{t} \ast \pure{\! P(t)}} \vdash \hoare{\TRUE}{e}{x.\; \textlog{lowval}\; x} \\
    (C[e]);(\emptyset, \textlog{OK}); \emptyset \longrightarrow^* T';(h',g'); t
  }
  {g' = \textlog{OK} \wedge P(t)}
\end{mathpar}

That is, if $e$ has been verified under the assumption that the predicate $P$
holds as a trace invariant, then executing $e$ in an adversarial context yields
a trace that does satisfy $P$.

$C \in \textit{AdvCtx}$ also has to be extended to forbid $C$ from containing
$\OutV$.

\subsection{Capability Machine with MMIO}

\newcommand{\VR}{\mathcal{V}}
\newcommand{\ER}{\mathcal{E}}
\newcommand{\RR}{\mathcal{R}}
\newcommand{\notMMIO}{\overline{\MMIO}}

Similarly to $\textlog{lift}$, we can generalize the existing logical relation
to thread a constraint on directly accessible memory locations. One would
parameterize the value, expression and register relations ($\VR$, $\ER$ and $\RR$)
with a predicate $\Psi$ on addresses.

Then, for any permission $p$ that includes either the R or W bit, $\VR$
is extended as follows:

\[
  \VR^{\Psi}(p, g, b, e, a) \eqdef \underbrace{\ldots}_{\text{as before}} \ast \; \pure{\forall a' \in [b,e).\; \Psi(a')}
\]

Then, $\ER^{\Psi}$ and $\RR^{\Psi}$ are simplify defined by threading the extra
$\Psi$ parameter through the existing definition.

Finally, for our relation to characterize ``low values'' that do not give direct
access to MMIO addresses, one would instantiate $\Psi$ with a predicate
$\notMMIO$ that excludes memory mapped addresses:

\[
  \notMMIO(a) \eqdef a \notin \MMIO
\]

Then, $\VR^{\notMMIO}$ is similar to the $\textlog{lowval}$ predicate of OCPL.
Intuitively, a value in the relation does not directly point to memory-mapped
locations, nor can it gain access to memory-mapped locations indirectly,
similarly to how $\textlog{lowval}$s cannot grant access to high locations
(in)directly. The relation $\ER^{\notMMIO}$ specifies that, if we fill the
registers with values in $\VR^{\notMMIO}$, then the
invariants in a private future world of our starting world will be satisfied at the end of execution.

\paragraph*{Remark:} As stated above, the ``generalized'' value relation is in
fact not very useful for instantiation of $\Psi$ other than $\notMMIO$. Indeed,
even if $\Psi$ \emph{does} allow referring to addresses in the $\MMIO$ region,
the value relation does not grant any corresponding resources.
The fix would be to use the generalized trace resources mentioned previously,
and grant ownership for the part of the trace corresponding to the
addresses in $[b,e) \cap \MMIO_{\X{pub}}$ with $\MMIO_{\X{pub}}$ the shareable subset of addresses in \MMIO.

We will have to adapt the robust safety theorem to this new capability machine
setting, \'and to the more concrete setup of the examples we want to gradually
verify.

Ignoring this second restriction for the time being, the general updated robust
safety theorem might look something like this:

\begin{mathpar}
  \mprset{vskip=0.4em}
  \inferrule
  { \forall a_{i} \in A_{\X{entry}} \ldotp \left\{
      {\begin{array}{ll}
      \knowInv{\iota}{\exists t.\; \tracefrag{t} \ast P(t)}\! \ast \MMIOag
         \vdash\\
         \quad \forall W \ldotp\;(\ER^{\notMMIO} (RX, g, b, e, a_i))\spac W
      \end{array}} \right. \\
     \\
    (r[\overline{r_{a_{i}} := (E, g, b, e, a_i)}], m, \emptyset) \longrightarrow^*
    (r', m', t)
  }
  {P(t)}
\end{mathpar}

The capability $(RX, g, b, e, a_i)$ points to the $i$th entry point of the code
we want to gradually verify.
The set of entry points is given by $A_{\X{entry}}$.
The capability $(E, g, b, e, a_i)$ then represents
a closure for this same piece of code. The adversary only gets access to the
closure, in order to avoid them from executing or reading arbitrary lines of
code within the trusted module.
Notice the universal quantification over worlds in the precondition;
this needs to be there, since we do not know in what world our code will be
called.
This universal quantification was not explicitly present in the OCPL
formalization.
The reason for that is that (I think so at least, do not quote me
on this - Thomas) they
used the built-in Iris worlds within their definition of weakest precondition,
by leveraging the state interpretation, which is universally quantified over in
the definition of weakest precondition.
As we discussed before, if we do away
with local capabilities, it should be possible to fall back onto Iris' notion of
worlds, which would allow us to remove the universal quantification in the
above definition too because it would be implicitly present in the WP inside $\ER$.

\paragraph*{Remark:} it would be interesting to have a discussion about how to
do away with the worlds if we remove local capabilities, regardless of whether
we will in the end, since that might shed some light on why the worlds look the
way they do right now.

Also note that we might as well have written
$\ldots \vdash \forall W \ldotp \; (\VR^{\notMMIO} \ldots $ %)
in the above theorem, instead of having the expression relation, since this is
the actual relation we will need, and easily unfolds to the former statement
through the definition of the value relation. I wrote $\ER$ to place the focus
on the execution of the driver being safe. I consider this a matter of taste.

The above theorem is still lacking, since it has no notion of adversarial
context \textit{AdvCtx} to limit the power of the attacker. Indeed, if we allow
an attacker to have read or write access to any location in \MMIO, or possess
any capability directly pointing into the driver, then all hope we might have of
gradually enforcing safety properties is now lost. We need a check on the memory
and registers that we will allow our driver closures to be used with.

To enforce these syntactic restrictions, we have two options:
\begin{enumerate}
\item Add a syntactic check to the statement of our robust safety theorem,
  ensuring that the adversary code cannot gain access to any non-E capabilities
  pointing into the driver code or any capabilities pointing into $\MMIO$ directly.
\item Assume the existence of a piece of boot code, that gets to run first when
  the capability machine starts up, and makes sure the above condition actually
  holds before passing control to the adversary. We could either assume that the
  adversarial memory can contain any capability at startup, in which case the
  driver has to erase the memory before passing control, or we could assume that
  the memory cannot contain any capabilities whatsoever, in which case the boot
  code does not have this obligation.
\end{enumerate}

If we were to follow option 2 and write concrete boot code, part of our proof
would consist of reducing option 2 to the situation in option 1, since this is
where unknown adversary code actually starts executing. In a way, option 1 is
hence more fundamental than option 2. We therefore adapt the robust safety
theorem to option 1 first.

In this setting, we can subdivide the memory $m$ into 3 relevant subsections;
$m = m_{\MMIO} \uplus m_{\X{driver}} \uplus m_{\X{adv}}$ %\uplus m_{\X{frame}}
with the following meaning:
\begin{itemize}
\item $m_{\MMIO}$ contains the memory-mapped locations
\item $m_{\X{driver}}$ contains the driver code
\item  $m_{\X{adv}}$ contains the adversary's code and data (this will also
  include the boot code in the general setting, since it does not contain any
  inherently dangerous capabilities itself)
\end{itemize}.

\newcommand{\notMapsToR}[2]{\rightarrow^{\overline{#1}}\!\!(#2)}

We can then implement the syntactic check using the following predicate
$\notMapsToR{R}{w}$, that takes a word, memory region or register file and checks if it points into the forbidden region $R$ as follows:

\[
  \begin{array}{lcl}
    \notMapsToR{R}{\X{inl}\spac z} & \eqdef
    & \TRUE \\
    %
    \notMapsToR{R}{\X{inr}\spac (p,g,b,e,a)} & \eqdef & R \cap [b,e[\; =
                                                 \emptyset  \\
    %
    \notMapsToR{R}{\V{reg}} & \eqdef & \forall r \in \dom(\V{reg}) \ldotp\spac \notMapsToR{R}{\V{reg}(r)} \\
    %
    \notMapsToR{R}{\V{mem}} & \eqdef & \forall l \in \dom(\V{mem}) \ldotp\spac   \notMapsToR{R}{\V{mem}(l)}
  \end{array}
\]

Given this definition, we can now formalize the initialization constraints on
registers $r$ and memory $m$, given the region $\MMIO$ and
the driver's address space $[b,e[$ as follows;

\begin{mathpar}
  \mprset{vskip=0.4em}
  \inferrule
  { m = m_{\MMIO} \uplus m_{\X{driver}} \uplus m_{\X{adv}}\\
    m_{\MMIO} = \MMIO \rightarrow \mbox{\Lightning} \\
    \dom( m_{\X{driver}} ) = [b,e[\\
    \dom( m_{\X{adv}} ) = [b_{\X{adv}},e_{\X{adv}}[\\
    \notMapsToR{\MMIO \cup [b,e[}{\V{r}}\\
    \notMapsToR{\MMIO \cup [b,e[}{\V{m_{\X{adv}}}} \\
    r \regdr \X{PC} = (\X{RWX}, \X{G},b_{\X{adv}},e_{\X{adv}},b_{\X{adv}})\\
  }
  {\initOK{\MMIO}{[b,e[}}
\end{mathpar}


and subsequently formulate the robust safety theorem as follows:

\begin{mathpar}
  \mprset{vskip=0.4em}
  \inferrule
  { \forall a_{i} \in A_{\X{entry}} \ldotp \left\{
      {\begin{array}{ll}
      \knowInv{\iota}{\exists t.\; \tracefrag{t} \ast P(t)}\! \ast \MMIOag
         \vdash\\
         \quad \forall W \ldotp\;(\ER^{\notMMIO} (RX, g, b, e, a_i))\spac W
       \end{array}} \right. \\
   \initOK{\MMIO}{[b,e[} \\
  (r[\overline{r_{a_{i}} := (E, g, b, e, a_i)}], m, \emptyset) \longrightarrow^* (r', m', t)\\
  }
  {P(t)}
\end{mathpar}
, where we have set the PC up to be a sensible capability (pointing to $b_{\X{adv}}$
without loss of generality).

Notice how the syntactic check for the registers is only performed on the set of registers $r$, allowing the closures in $\overline{r_{a_i}}$ to point into to the driver address space.

Having this theorem in our toolbox, we can now take a closer look at scenario 2
above, where a piece of boot code is used to set the memory up correctly,
obviating the need for any other assumptions than that the boot code behaves
according to some spec. Concretely, we want the boot code to uphold a spec that
allows us to prove the above theorem.
Alternatively, we could also just implement one specific instance of boot code, and prove that after it finishes executing, all preconditions to apply the above theorem are met.
We can use the following predicate to specify that the starting configuration is
valid:

\begin{mathpar}
  \mprset{vskip=0.4em}
  \inferrule
  { m = m_{\MMIO} \uplus m_{\X{driver}} \uplus m_{\X{boot}}\\
    m_{\MMIO} = \MMIO \rightarrow \mbox{\Lightning} \\
    \dom(m_{\X{driver}}) = [b,e[\\
    \dom(m) = [0,\X{MEM_{MAX}}[\\
    r \regdr \X{PC} = (\X{RWX}, \X{G},0,\X{MEM_{MAX}},l_{\X{boot}})\\
  }
  {\initBC{\MMIO}{[b,e[}}
\end{mathpar}
where $l_{\X{boot}}$ is the location where the machine boots, to be determined
by the machines operational semantics.

On the level of the operational semantics, we have to prove that the
following holds, in order to use our concrete driver implementation with the
robust safety theorem above (the driver still needs to be proven safe separately):

\begin{mathpar}
  \mprset{vskip=0.4em}
  \inferrule
  { \initBC{\MMIO}{[b,e[}\\
  }
  { \exists n\ldotp (r, m, \emptyset) \longrightarrow^n (r'[\overline{r_{a_{i}}
      := (E, g, b, e, a_i)}], m', t) \land P(t) \land\\
  \initOK{\MMIO}{[b,e[}[r'][m']}\quad (\textsc { BootCodeOK })
\end{mathpar}

Note that to link scenario 2 to scenario 1, we need a slightly adapted version of the robust safety theorem stated above
where the initial trace is allowed to not be empty, but rather just satisfy
$P(t)$, but that should not cause any problems.
For a specific instance of boot code that provably satisfies the \textsc{BootCodeOK}
rule, the resulting boot-code-based formulation of the robust safety theorem is now
just a corollary of the general robust safety theorem, and has the following
statement:

\begin{mathpar}
  \mprset{vskip=0.4em}
  \inferrule
  { \forall a_{i} \in A_{\X{entry}} \ldotp \left\{
      {\begin{array}{ll}
      \knowInv{\iota}{\exists t.\; \tracefrag{t} \ast P(t)}\! \ast \MMIOag
         \vdash\\
         \quad \forall W \ldotp\;(\ER^{\notMMIO} (RX, g, b, e, a_i))\spac W
       \end{array}} \right. \\
   \initBC{\MMIO}{[b,e[} \\
  (r, m, \emptyset) \longrightarrow^* (r', m', t)\\
  }
  {P(t)}
\end{mathpar}

Now we have a closer look at the concrete example we would like to verify,
including what its boot code looks like. Our goal is to prove that this concrete
example satisfies \textsc{BootCodeOk} and this last robust safety theorem.

The code for for the driver scenario we want to gradually verify can
be found in \T{driver\_code.v}. This example file contains more details on the
scenario in the comments. From a more high-level perspective, it contains a
trusted part, consisting of a single MMIO location, boot code to properly set up
the capability machine at start-up, and the code for the driver itself. On the
other hand, the example also contains an untrusted part, consisting of a known,
but untrusted code section and sandbox section (which we do not make assumptions
over).
The first address of the adversary code section is what the trusted part jumps
to once it has finished setting up.

The boot code is where execution starts off when the capability machine is
powered on.
It contains an omnipotent RWX capability (i.e. ranging over all of memory and
providing full permissions over it) starts off execution.
(\textbf{Design Alternative:} in the long run, it might be worthwhile to provide
the boot code with a RWLX capability. Since RWX cannot be upgraded to RWLX, it
is impossible in the current setting to develop a secure stack calling
convention \'a la Lau, where the stack capability has to be local-WL and (in this
case) the driver code has to ensure that there is no other WL than the stack.
However, having a global stack capability is currently not an issue, since the interface to our driver is first order, i.e. it will
never create another adversary stack frame on top of itself (currently, it does
not even use the stack in the first place!), and since we are working in a
single-threaded setting (not very relevant, but I think the multi-threaded
setting is an interesting thought experiment). I avoided unnecessary
complications and hence kept the omnipotent capability as RWX. If the omnipotent
capability were to become RWLX in the future, we would have to be careful using
a stack, since if we want our secure calling convention with a higher order
interface to our driver API, there cannot be any adversary-accessible global
write-local memory, i.e. the reduced omnipotent capability cannot be allowed to be global if we pass it to the adversary. We would thus have to pass a stack capability that we explicitly make local, and a global RWX capability for the rest of memory)

The boot code has the following responsibilities:
\begin{itemize}
\item Generate the necessary closures (i.e. Enter capabilities) for the driver's read and write methods, so that the adversary can use safely use them to perform I/O, and any properties that the driver wants to uphold on the input-\textbackslash output stream can actually be enforced.
\item Wipes all of the adversary's sandbox section, to make sure no remaining
  capabilities can be found there (\textbf{Design Alternative:} maybe we should try to find out how CHERI handles this; is it possible that any lingering capabilities are left in RAM
  on machine start-up? In any case, we are already being slightly unrealistic by assuming the adversary code section is just \emph{there}. We have the option
  to not do any erasure, and just make the assumption that adversary memory
  contains no capabilities pointing into MMIO or pointing into the driver. ).
  It does n\'ot, however, wipe the untrusted, hard-coded adversary routine that reads $N$ lines of code from a single (and the only) MMIO location in memory.
\item Reduces the omnipotent capability to provide RWX access to all of the
  adversary's code, makes it point to the first element of the adversary's code
  section and jumps to it, in order to hand control to the adversary. Before the
  jump, it clears all registers except for the 2 containing the read and write
  driver closures, in order to make sure that no extra permissions leak to the adversary.
\end{itemize}

\bibliographystyle{alpha}
\bibliography{biblio}

\appendix

\end{document}
