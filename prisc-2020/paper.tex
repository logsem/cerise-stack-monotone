%% For double-blind review submission, w/o CCS and ACM Reference (max submission space)
\documentclass[sigplan,review]{acmart}\settopmatter{printfolios=true,printccs=false,printacmref=false}
%% For double-blind review submission, w/ CCS and ACM Reference
%\documentclass[sigplan,review,anonymous]{acmart}\settopmatter{printfolios=true}
%% For single-blind review submission, w/o CCS and ACM Reference (max submission space)
%\documentclass[sigplan,review]{acmart}\settopmatter{printfolios=true,printccs=false,printacmref=false}
%% For single-blind review submission, w/ CCS and ACM Reference
%\documentclass[sigplan,review]{acmart}\settopmatter{printfolios=true}
%% For final camera-ready submission, w/ required CCS and ACM Reference
%\documentclass[sigplan]{acmart}\settopmatter{}


%% Conference information
%% Supplied to authors by publisher for camera-ready submission;
%% use defaults for review submission.
\acmConference[PL'18]{ACM SIGPLAN Conference on Programming Languages}{January 01--03, 2018}{New York, NY, USA}
\acmYear{2018}
\acmISBN{} % \acmISBN{978-x-xxxx-xxxx-x/YY/MM}
\acmDOI{} % \acmDOI{10.1145/nnnnnnn.nnnnnnn}
\startPage{1}

%% Copyright information
%% Supplied to authors (based on authors' rights management selection;
%% see authors.acm.org) by publisher for camera-ready submission;
%% use 'none' for review submission.
\setcopyright{none}
%\setcopyright{acmcopyright}
%\setcopyright{acmlicensed}
%\setcopyright{rightsretained}
%\copyrightyear{2018}           %% If different from \acmYear

%% Bibliography style
\bibliographystyle{ACM-Reference-Format}
%% Citation style
%\citestyle{acmauthoryear}  %% For author/year citations
%\citestyle{acmnumeric}     %% For numeric citations
%\setcitestyle{nosort}      %% With 'acmnumeric', to disable automatic
                            %% sorting of references within a single citation;
                            %% e.g., \cite{Smith99,Carpenter05,Baker12}
                            %% rendered as [14,5,2] rather than [2,5,14].
%\setcitesyle{nocompress}   %% With 'acmnumeric', to disable automatic
                            %% compression of sequential references within a
                            %% single citation;
                            %% e.g., \cite{Baker12,Baker14,Baker16}
                            %% rendered as [2,3,4] rather than [2-4].


%%%%%%%%%%%%%%%%%%%%%%%%%%%%%%%%%%%%%%%%%%%%%%%%%%%%%%%%%%%%%%%%%%%%%%
%% Note: Authors migrating a paper from traditional SIGPLAN
%% proceedings format to PACMPL format must update the
%% '\documentclass' and topmatter commands above; see
%% 'acmart-pacmpl-template.tex'.
%%%%%%%%%%%%%%%%%%%%%%%%%%%%%%%%%%%%%%%%%%%%%%%%%%%%%%%%%%%%%%%%%%%%%%


%% Some recommended packages.
\usepackage{booktabs}   %% For formal tables:
                        %% http://ctan.org/pkg/booktabs
\usepackage{subcaption} %% For complex figures with subfigures/subcaptions
                        %% http://ctan.org/pkg/subcaption
%\usepackage[showframe]{geometry}% http://ctan.org/pkg/geometry
\usepackage[utf8]{inputenc}
\usepackage{amsmath}
\usepackage{amssymb}
\usepackage{tikz}
\usetikzlibrary{decorations.pathreplacing}
\usepackage{xcolor}
\usepackage{graphicx} % need this package
\usepackage{stmaryrd}
\usepackage{amsthm}
\usepackage{dsfont}
\usepackage{float}
\usepackage{IEEEtrantools}
\usepackage[colorinlistoftodos,prependcaption,textsize=tiny]{todonotes}
\presetkeys%
    {todonotes}%
    {inline,backgroundcolor=orange}{}

\newcommand{\sep}{-\kern-.6em\raisebox{-.659ex}{*}\ }
\newcommand{\bupd}[1]{=\kern-.6em\{#1\}\ \kern-.9em =\ \kern-.8em\raisebox{-.659ex}{*}\ }

\newcommand{\bigsep}{\mathop{\scalebox{2.5}{\raisebox{-0.4ex}{$*$}}}}%

\newcommand{\ra}[1] {
  \begin{tikzpicture}
        \node[draw,dashed] {#1};
    \end{tikzpicture}}
    
\newcommand{\inv}[1] {
  \begin{tikzpicture}
        \node[draw] {#1};
    \end{tikzpicture}}
    
\newcommand{\interp}[2]{(#1)(#2)}
\newcommand\sepimp{\mathrel{-\mkern-6mu*}}

\usetikzlibrary{decorations.pathreplacing}

\newcommand\MemoryLayout[1]{
  \begin{tikzpicture}[scale=0.3]
     \draw[thick](0,0)--++(0,1);
     \foreach \pt/\col/\lab [remember=\pt as \tp (initially 0)] in {#1} {
     \if\lab\relax\relax\else
         \draw[thick,decorate, decoration={brace,amplitude=4mm}]
            (-\tp,-0.2)--node[below=4mm]{\lab} (-\pt+2,-0.2);
            \draw[fill=\col] (-\tp,0) rectangle (-\pt+2,1);
       \fi
       \foreach \a in {\tp,...,\pt} {
          \draw(-\a,0) rectangle ++(-1,1);
       }
       \draw[thick](-\pt,0)--++(0,1);
       
     }
     \node at (-4.5,0.6) {\small B};
  \end{tikzpicture}
}

\def\world{\emph{World}}
\def\word{\emph{Word}}

\begin{document}

%% Title information
\title[Short Title]{Mechanized Reasoning about a Capability Machine}
%% [Short Title] is optional;
                                        %% when present, will be used in
                                        %% header instead of Full Title.
%\titlenote{with title note}             %% \titlenote is optional;
                                        %% can be repeated if necessary;
                                        %% contents suppressed with 'anonymous'
%\subtitle{Subtitle}                     %% \subtitle is optional
%\subtitlenote{with subtitle note}       %% \subtitlenote is optional;
                                        %% can be repeated if necessary;
                                        %% contents suppressed with 'anonymous'


%% Author information
%% Contents and number of authors suppressed with 'anonymous'.
%% Each author should be introduced by \author, followed by
%% \authornote (optional), \orcid (optional), \affiliation, and
%% \email.
%% An author may have multiple affiliations and/or emails; repeat the
%% appropriate command.
%% Many elements are not rendered, but should be provided for metadata
%% extraction tools.

%% Author with single affiliation.
\author{A{\"i}na Linn Georges}
%\authornote{with author1 note}          %% \authornote is optional;
                                        %% can be repeated if necessary
%\orcid{nnnn-nnnn-nnnn-nnnn}             %% \orcid is optional
\affiliation{
  %\position{Position1}
  %\department{Department1}              %% \department is recommended
  \institution{Aarhus University}            %% \institution is required
  %\streetaddress{Street1 Address1}
  %\city{City1}
  %\state{State1}
  %\postcode{Post-Code1}
  %\country{Denmark}                    %% \country is recommended
}
\email{ageorges@cs.au.dk}          %% \email is recommended

\author{Alix Trieu}
%\authornote{with author2 note}          %% \authornote is optional;
                                        %% can be repeated if necessary
%\orcid{nnnn-nnnn-nnnn-nnnn}             %% \orcid is optional
\affiliation{
  %\position{Position2a}
  %\department{Department2a}             %% \department is recommended
  \institution{Aarhus University}           %% \institution is required
  %\streetaddress{Street2a Address2a}
  %\city{City2a}
  %\state{State2a}
  %\postcode{Post-Code2a}
  %\country{Denmark}                   %% \country is recommended
}
\email{alix.trieu@cs.au.dk}         %% \email is recommended


\author{Lars Birkedal}
%\authornote{with author2 note}          %% \authornote is optional;
                                        %% can be repeated if necessary
%\orcid{nnnn-nnnn-nnnn-nnnn}             %% \orcid is optional
\affiliation{
  %\position{Position2a}
  %\department{Department2a}             %% \department is recommended
  \institution{Aarhus University}           %% \institution is required
  %\streetaddress{Street2a Address2a}
  %\city{City2a}
  %\state{State2a}
  %\postcode{Post-Code2a}
  %\country{Denmark}                   %% \country is recommended
}
\email{birkedal@cs.au.dk}         %% \email is recommended



%% Abstract
%% Note: \begin{abstract}...\end{abstract} environment must come
%% before \maketitle command
\begin{abstract}
  Capability machines are promising targets for secure compilers since capabilities can be used to enforce abstractions that are usually expected for high-level languages, such as well-bracketed control-flow (WBCF) and local state encapsulation (LSE).
  We present the first formalization of a capability machine that supports mechanized reasoning about deep semantic properties, including WBCF and LSE.
  Our formalization is done in the Coq implementation of Iris, a state-of-the-art concurrent higher-order separation logic, and includes a formalization of the logical relation defined by Skorstensgaard et. al. \cite{skorstengaardESOP18}, which can used to prove WBCF and LSE.
\end{abstract}


%% 2012 ACM Computing Classification System (CSS) concepts
%% Generate at 'http://dl.acm.org/ccs/ccs.cfm'.
\begin{CCSXML}
<ccs2012>
<concept>
<concept_id>10011007.10011006.10011008</concept_id>
<concept_desc>Software and its engineering~General programming languages</concept_desc>
<concept_significance>500</concept_significance>
</concept>
<concept>
<concept_id>10003456.10003457.10003521.10003525</concept_id>
<concept_desc>Social and professional topics~History of programming languages</concept_desc>
<concept_significance>300</concept_significance>
</concept>
</ccs2012>
\end{CCSXML}

\ccsdesc[500]{Software and its engineering~General programming languages}
\ccsdesc[300]{Social and professional topics~History of programming languages}
%% End of generated code


%% Keywords
%% comma separated list
\keywords{keyword1, keyword2, keyword3}  %% \keywords are mandatory in final camera-ready submission


%% \maketitle
%% Note: \maketitle command must come after title commands, author
%% commands, abstract environment, Computing Classification System
%% environment and commands, and keywords command.
\maketitle


\section{Introduction}

Capability machines allow for fine grained control over the authority of memory \cite{levy_capability-based_1984,watson_cheri:_2015}.
At the machine level, pointers are replaced by capabilities, to which is attached a range of authority and a permission.
When the machine executes an instruction it dynamically checks that the instruction uses a capability within its range of authority.
Capability machines are promising targets for secure compilers because these dynamic checks can be used to enforce abstractions that are usually expected for high-level languages, such as well-bracketed
control-flow (WBCF) and local state encapsulation (LSE). 

We present the first formalization of a capability machine that supports mechanized reasoning about deep semantic properties, including WBCF and LSE.

Our formalization builds upon earlier work of \citeauthor{skorstengaardESOP18} \cite{skorstengaardESOP18} and \cite{SkorstengaardPOPL19}, who present two different capability machines and calling conventions that enforce well-bracketed control flow, and methods for defining and reasoning about capability machines.
In each case, they define a logical relation to capture a semantic
notion of capability safety and use it to prove WBCF and LSE.

The logical relations in \cite{skorstengaardESOP18} and \cite{SkorstengaardPOPL19} are so-called step-indexed Kripke logical relations, which means that they are indexed over recursively-defined worlds, which contain descriptions of invariants of the memory of the machine.
It is well-known that it is non-trivial to define and work with such step-indexed Kripke logical relations~\cite{ahmed-2002, birkedal2011step}.
Therefore we do not formalize the logical relations of \citeauthor{skorstengaardESOP18} 
directly, but rather give a more abstract \emph{logical} definition of the logical relations in the Iris program logic framework \cite{iris,iris2,iris3, iris-jfp}, which comes with built-in support for abstract reasoning about recursion (qua the later modality and L{\"o}b induction) and invariants.
Such a logical approach to defining logical relations has been used successfully before for logical relations for typed high-level languages (e.g., \cite{lslr-journal,iris:type-effect,rustbelt,runst-popl18}); here we use it for the first time for a low-level untyped machine language.
Another reason for using Iris is that we can use the Coq implementation of Iris and
the Iris proof mode \cite{Krebbers:2017} to mechanize our development. 

In summary, we present Iris formalizations of 
\begin{itemize}
  \item a program logic for reasoning about capability machine programs.
  \item the logical relation from \cite{skorstengaardESOP18}, 
     which captures capability safety and which can be used to reason about WBCF and LSE.
\end{itemize}
All of the technical development is mechanized in Coq using the Coq implementation of Iris.

% Such a logical approach to defining logical relations has been used successfully before for logical relations for typed high-level languages (e.g., \cite{lslr, XXX}); here we use it for the first time for a low-level untyped machine language.
% Such a logical approach to defining logical relations has been used successfully before for logical relations for typed high-level languages (e.g., \cite{lslr, XXX}); here we use it for the first time for a low-level untyped machine language.


% In \cite{skorstengaardESOP18} and \cite{SkorstengaardPOPL19}, Skorstensgaard et. al. present two different calling conventions that enforce well bracketed control flow, and methods for defining and reasoning about capability machines. In each case, they define a step-indexed Kripke logical relation to define a notion of capability safety. 

% However, such logical relations are notoriously difficult to construct, due to the so-called World circularity problem \cite{Ahmed2004SemanticsOT}. This challenge has been addressed by giving a logical account of the relation, thus raising the level of abstraction 
% \cite{Krogh-Jespersen2017} \cite{dreyer_neis_birkedal_2012}. Such logical accounts have typically been for logical relations of typed high level languages. We show how to give a logical account of an untyped low level language. 

% Iris is a higher-order concurrent separation logic framework. Its support of higher-order ghost state makes Iris a great candidate for complex reasoning about \textit{state}. This is exactly what we need for reasoning about a machine, where computation consist of state manipulation. Furthermore, the model of Iris is itself step-indexed, and comes equipped with a later modality: $\triangleright$. The later modality, and other logical features of Iris, is enough to define these step-indexed logical relations\cite{Timany:2017} \cite{Jung:2017} \cite{timany2019}. Additionally, we can take advantage of the Iris proof mode \cite{Krebbers:2017} for an interactive proof style in Coq. 

% We have formalized the unary logical relation defined in \cite{skorstengaardESOP18} in Iris. The formalization makes the following contributions: 
% \begin{itemize}
% 	\item A formalization of a capability machine program logic in Iris, in which programs lie in memory, and execution is completely stateful. 
% 	\item An Iris formalization of a Logical Relation to reason about programs in a capability machine language, which can be used to reason about local state while distinguishing between well-bracketed, and non well-bracketed calls.
% \end{itemize} 


\section{The Program Logic of a Capability Machine}

\begin{itemize}
\item we define the operational semantics of the CM
\item we prove basic hoare triples for each of the atomic instructions
  of the CM (i.e., for load, store, etc.)
\item non-trivial
  \begin{itemize}
  \item while Iris is a framework and supports many languages, it is
    geared towards models of higher-level languages which abstract from
    program counters and the fact that programs are stored in memory.
    -- here our low-level machine model includes registers, in particular
    the program counter register, and instructions are stored in memory.
  \item to capture the semantics of the CM in the Iris framework, 
    we introduce an abstract notion of instruction, whose
    operational meaning is to execute the next instruction the program counter
    points to, and abstract notions of values, failed and halted.
  \item for each machine instruction, we can then prove a hoare triple
    which expresses the pre- and post-conditions for executing the instruction
  \end{itemize}
\end{itemize}

In order to give a logical account of the logical relation, we must first define a program logic for the capability machine language. Iris is a separation logic framework, which means we can embed any language into Iris by describing its small step semantics. 

While Iris is a framework and supports many languages, it is geared towards models of higher-level languages. Embedding a language asks for a definition of expressions, values, and the state, which abstracts away from the fact that programs are stored in memory. 

Typically a machine will have a special register; the program counter, which contains a pointer to an address in memory. That address in turn will contain an integer, which can then be decoded to various machine dependent instructions, such as Load, Store, Jump, etc. Once an instruction has been executed, the program counter is incremented and will now point to the next instruction in memory. 

To capture the semantics of the Capability Machine in the Iris framework, we introduce an abstract notion of an instruction, whose operational meaning is to execute the instruction the program counter points to, and an abstraction notion of values for halted and failed configurations.

Next we use the Iris framework to prove Hoare Triples for each instruction. Like in \cite{Jensen:2013}, we define a notion of points-to for registers. Since a capability machine replaces pointers with capabilities, we will also replace the conventional points-to for memory with a points-to assertion with a permission attached to it: 
$$ r \mapsto_r w \hspace{1cm} a \mapsto_a[p]~w $$
The capability points-to predicate states that address $a$ points to word $w$ with permission $p$. The permission restricts the words an address can update to. For instance, in $a \mapsto_a[\textsc{\tiny RO}]~w$, the $\textsc{\tiny RO}$ permission does not allow for any updates to $a$. 

\todo{Alix: We don't really see the weakest precondition in this example, shouldn't we say Hoare triple instead?}
Similar to \cite{Jensen:2013}, the program logic rules for each instruction are of the following form: 
\begin{align*}
\small
	\textit{run time conditions } \wedge &\textit{ decode}(w) = instr \Rightarrow  \\
	\{\{\{ \textsf{PC} \mapsto_r ((p,g),b,e,a) &* a \mapsto_a[p] w * ... \}\}\} \\ 
	\textsf{Instr Executable }\\
	\{\{\{\textsf{PC} \mapsto_r ((p,g),b,e,a&+1) * a \mapsto_a[p] w * ... \}\}\}
\end{align*} 
Where the run time conditions correspond to the dynamic checks done by the capability machine, and \textsf{Instr Executable} is the abstract expression \todo{Alix: (an artificial expression)} for executing the next instruction in memory. Unlike \cite{Jensen:2013}, the decoding function is in our case assumed. We now have the necessary building blocks for implementing the logical relation. 

\section{Logical Relation}


\begin{figure*}[ht]
	\begin{minipage}{\textwidth}
	\begin{align*}
		\mathcal{E}\interp{W}{pc} \triangleq&~\forall r, \mathcal{R}(W)(r) ~*~\textsf{context}(W)(r[\textsf{\small PC}:=pc])\\
		&\sep~\textsc{WP}~\textsf{Seq (Instr Executable)}~\{ v, v = HaltedV \implies \exists W' r', W' \sqsubseteq_{priv} W * \textsf{context}(W')(r')\}
	\end{align*}
	\end{minipage}
	\caption{Logical Relation for Expressions}
	\label{fig:expr}
\end{figure*}

\begin{figure*}[ht]
	\begin{minipage}{\textwidth}
	\begin{align*}
		\mathcal{R}\interp{W}{r} \triangleq&~\forall (reg : \text{RegName} \backslash \textsf{PC}), \mathcal{V}\interp{W}{r(reg)}
	\end{align*}
	\end{minipage}
	\caption{Logical Relation for Register States}
	\label{fig:regs}
\end{figure*}

\begin{figure*}[ht]
	\begin{minipage}{\textwidth}
	\begin{align*}
 		\mathcal{V}\interp{W}{z} \triangleq&~\exists z' \in \mathds{Z}. z = z' \\
 		\mathcal{V}\interp{W}{((\text{\tiny{O}},g),b,e,a)} \triangleq&~\top
 \\
 		\left.
		\begin{IEEEeqnarraybox}[\IEEEeqnarraystrutmode \IEEEeqnarraystrutsizeadd{2pt}{2pt}][c]{rCl} 
			\mathcal{V}\interp{W}{((\text{\tiny{RO}},g),b,e,a)} \\
			\mathcal{V}\interp{W}{((\text{\tiny{RW}},g),b,e,a)} \\
			\mathcal{V}\interp{W}{((\text{\tiny{RWL}},g),b,e,a)}
  \end{IEEEeqnarraybox}
  \, \right\}  &\triangleq  \left.\exists~p', p \sqsubseteq p' * \textsf{read\_write\_cond}(\textsf{p',b,e})
  \right. &\textit{where p is {\tiny{RO}}, {\tiny{RW}}, {\tiny{RWL}} resp.}
  \\
   		\left.
		\begin{IEEEeqnarraybox}[\IEEEeqnarraystrutmode \IEEEeqnarraystrutsizeadd{2pt}{2pt}][c]{rCl} 
			\mathcal{V}\interp{W}{((\text{\tiny{RX}},g),b,e,a)} \\
			\mathcal{V}\interp{W}{((\text{\tiny{RWX}},g),b,e,a)} \\
			\mathcal{V}\interp{W}{((\text{\tiny{RWLX}},g),b,e,a)}
  \end{IEEEeqnarraybox}
  \, \right\}  &\triangleq  \left.
	\begin{IEEEeqnarraybox}[
    \IEEEeqnarraystrutmode][c]{rCl}
    \exists~p', p \sqsubseteq p' &*& \textsf{read\_write\_cond}(\textsf{p',b,e})\\
  &*& ~\square~\textsf{exec\_cond(W)(\textsf{p,g,b,e})}
  \end{IEEEeqnarraybox}  
  \right. &\textit{where p is {\tiny{RX}}, {\tiny{RWX}}, {\tiny{RWLX}} resp.}
  \\
 		\mathcal{V}\interp{W}{((\text{\tiny{E}},g),b,e,a)} \triangleq&~\square~\textsf{enter\_cond}(W)(\textsf{g,b,e,a})
	\end{align*}
	\end{minipage}
	\caption{Logical Relation for Words}
	\label{fig:relation}
\end{figure*}


\begin{figure*}[ht]
	\begin{minipage}{\textwidth}
	\begin{align*}
 		\textsf{read\_write\_cond}(p,b,e) \triangleq&~\bigsep_{a \in [b,e]} \text{rel}(a,p,\mathcal{V})\\
 		\textsf{exec\_cond(W)(p,g,b,e)} \triangleq&~\forall a \in [b~e], W' \sqsubseteq^g W.~\rhd~\mathcal{E}\interp{W'}{((p,g),b,e,a)}\\
 		\textsf{enter\_cond(W)(g,b,e,a)} \triangleq&~\forall W' \sqsubseteq^g W.~\rhd~\mathcal{E}\interp{W}{((\textsf{\tiny{RX}},g),b,e,a)}
	\end{align*}
	\end{minipage}
	\\[2em]
	where $W' \sqsubseteq^{\tiny Global} W \iff W' \sqsubseteq_{priv} W$\\ and 
	$W' \sqsubseteq^{\tiny Local} W \iff W' \sqsubseteq_{pub} W$	
	
	\caption{Capability Conditions}
	\label{fig:condition}
\end{figure*}

\subsection{The Value Relation}
\begin{itemize}
	\item definition of the relation and World
	\item LS definitions
	\item now: standard islands are not used 
\end{itemize}

The logical relation is a unary relation that relates a word to a notion of \textit{capability safety}. The value relation $\mathcal{V}$ is an Iris relation of type $World \to Word \to iProp$, where the World is a collection of state transition systems used to reason about local state. 

This notion of World is different from the logical relation given by \cite{skorstengaardESOP18}, where the World is a collection of invariants describing the behaviour of all memory. Skorstensgaard et. al. use the World to define validity of capabilities, by relating the invariant of the region in question to certain standard islands for capabilities. 

In the logical account, we use the capability points-to assertion to both describe what we may get from the read permission of a capability, and the restrictions imposed on us by the write permission of a capability. On the one hand, a points-to assertion lets us execute an instruction using that location. On the other, the permission of the points-to assertion restricts which updates are allowed, and thus which instructions will fail. The permission of the points-to assertion suffices to describe the behaviour of the above mentioned standard islands. 

We want the value relation to give us the points-to predicate, while asserting that what the predicate points to is valid. The challenge is now to define how that should be done.

\subsection{Reasoning about Local State}
\begin{itemize}
	\item How to reason about local state in high level languages
	\item public/private distinction
	\item TT style
	\item difficulties w/ iris invariants 
\end{itemize}

Logical accounts of relations for high level languages with references have used Iris invariants to define the validity of a reference. In order to reason about local state, it suffices to add a future world relation only in the definition of the expression relation. This approach works in a high level language, where the calls are inherently well-bracketed.

In our case however, calls are not always well-bracketed. \citeauthor{skorstengaardESOP18}. distinguishes between well-bracketed and non well-bracketed calls by respectively using public- and private future world relations given the appropriate conditions. 

Iris does not make such a distinction in its implementation of invariants. Using invariants alone, is therefore not sufficient for our purposes. Instead we explicitly define notions of public and private future world relations, and use these notions to distinguish between well-bracketed and non well-bracketed calls. 

A callback is an executable capability. This distinction must therefore be done by the value relation. However, if the value relation is indexed by a world, we again have difficulties using Iris invariants. Asserting $\exists w, \mathcal{V}(W)(w)$ as invariant does not allow for $W$ to change. 

The difficulty is to support reasoning about local state, while distinguishing between public- and private future world relations in the value relation, rather than the expression relation. We implement this by using Iris' saved predicates, which allows us to save a predicate by associating it to a unique name. Using that unique name, we are then able to refer to that predicate somewhere else, and state it with the argument that we want. That way, if we save the value relation predicate and associate it to a location, we can later on state that predicate to the future world in question. 

%% Acknowledgments
\begin{acks}                            %% acks environment is optional
                                        %% contents suppressed with 'anonymous'
  %% Commands \grantsponsor{<sponsorID>}{<name>}{<url>} and
  %% \grantnum[<url>]{<sponsorID>}{<number>} should be used to
  %% acknowledge financial support and will be used by metadata
  %% extraction tools.

  This research was supported in part by the ModuRes Sapere Aude Advanced Grant from The Danish Council for Independent Research for the Natural Sciences (FNU) and by a Villum Investigator grant (no.
  25804), Center for Basic Research in Program Verification (CPV), from the VILLUM Foundation.
  
  % This material is based upon work supported by the
  % \grantsponsor{GS100000001}{National Science
  %   Foundation}{http://dx.doi.org/10.13039/100000001} under Grant
  % No.~\grantnum{GS100000001}{nnnnnnn} and Grant
  % No.~\grantnum{GS100000001}{mmmmmmm}.  Any opinions, findings, and
  % conclusions or recommendations expressed in this material are those
  % of the author and do not necessarily reflect the views of the
  % National Science Foundation.
\end{acks}


% Bibliography
\bibliography{reading_list}


%% Appendix
\appendix
\section{Appendix}

Text of appendix \ldots

\end{document}
