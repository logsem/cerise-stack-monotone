\documentclass{article}
\usepackage[utf8]{inputenc}
\usepackage{hyperref}
\usepackage{amsmath}
\usepackage{amssymb}
\usepackage{mathpartir}
\usepackage{xspace}
\usepackage[dvipsnames]{xcolor}
\usepackage{iris}

\setlength{\parskip}{0.3em}

\newcommand{\Z}{\mathbb{Z}}
\newcommand{\X}[1]{\ensuremath{\textrm{#1}}}
\newcommand{\I}[1]{\ensuremath{\texttt{#1}}}
\newcommand{\SL}{Separation Logic\xspace}
\newcommand{\pure}[1]{\left\lceil #1 \right\rceil}

\newcommand{\FIXME}[1]{{\color{MidnightBlue} FIXME: #1}}

\newcommand{\MMIO}{\textlog{MMIO}}

\title{Capabilities, MMIO and Robust Safety}
\date{March 10, 2020}

\begin{document}

\maketitle

\section{Memory Mapped I/O: Operational Semantics}

The proposal is to simply represent read and writes to memory-mapped IO
addresses as events in a trace. This says nothing a priori about devices that
might be connected to these IO regions. In particular, without additional
assumptions, reading a byte from a memory-mapped region just returns an
arbitrary value.

If at some point we want to reason under the assumption that we are connected to
a specific device that reacts in a specific way, we can express that as an extra
\SL assertion, that we assume as a pre-condition, and that restricts the set of
different traces that we might observe.

\[
  \begin{array}{lcl}
    \X{EventTy} & := & \X{IOWrite} \; | \; \X{IORead} \\
    \X{Event} & := & \X{EventTy} \times \X{Addr} \times \Z \\
    \X{Trace} & := & \X{list} \; \X{Event} \\
    \X{State} & := & \underbrace{\X{Reg} \times \X{Mem}}_{\text{old state}}
                     \times \X{Trace} \\
  \end{array}
\]

Values of type \X{State} represent the state of a configuration in the
small-step operational semantics. In this setup, we assume the whole semantics
to be parameterized by the range of memory-mapped addresses: \MMIO.

\[
  \begin{array}{lcl}
    \MMIO & := & [\MMIO_{b}, \MMIO_{e}) \\
  \end{array}
\]

\textit{NB:} An alternative presentation would be to make this range an
immutable part of \X{State}. It probably does not really matter in the end, we
should do whatever makes our life easier for instantiating Iris with the
semantics.

In the (current) operational semantics without MMIO, the operational semantics
of the \I{Load} instruction is:

\begin{mathpar}
  \inferrule[Load]
  { r[\X{src}] = (p,g,b,e,a) \\ \X{readAllowed}\; p \\ a \in [b,e) }
  { (r, m) \xrightarrow{\I{Load} \; \X{dst} \; \X{src}} (r[\X{dst} := m[a]], m) }
\end{mathpar}

With MMIO, we obtain two rules:

\begin{mathpar}
  \inferrule[MemLoad]
  { r[\X{src}] = (p,g,b,e,a) \\ \X{readAllowed}\; p \\ a \in [b,e) \\ a \notin \MMIO}
  { (r, m, t) \xrightarrow{\I{Load} \; \X{dst} \; \X{src}}
    (r[\X{dst} := m[a]], m, t) }

  \inferrule[IOLoad]
  { r[\X{src}] = (p,g,b,e,a) \\ \X{readAllowed}\; p \\ a \in [b,e) \\ a \in \MMIO}
  { (r, m, t) \xrightarrow{\I{Load} \; \X{dst} \; \X{src}}
    (r[\X{dst} := x], m, (\X{IORead}, a, x) :: t) }
\end{mathpar}

Notice how in the second rule, we read an arbitrary integer $x$, and record it
in the trace.


\paragraph*{Remark:} With this presentation, the specification of the
``machine'' is very much decoupled from the model of the devices it might be
communicating with. This is a good thing, I think.

Nevertheless, one could consider an alternative presentation where the model of
the devices is more tightly integrated with the semantics of the machine. For
instance, one could make the operational semantics parameterized with the
devices' model, where each device is modeled as having some internal state, the
ability to react on reads or writes, or to perform an internal step. Then, the
operational semantics would either step the usual way, or whenever a device
steps.

I believe this would be somewhat similar to the semantics of I/O system calls
through a foreign function interfaces as formalized in
CakeML~\cite{cakeml-vstte17io}.

\paragraph*{Remark:} Alix says that the proposed style of operational semantics
(using a trace) is close to the semantics of \texttt{volatile} as in
CompCert---which is also how memory mapped addresses seem to be exposed to C
compilers in practice. So this is probably a good sign. 


\section{\SL Resources}

We added a trace as part of the state, so we wish to also expose it as a \SL
assertion.

\newcommand{\tracefull}[1]{\ownGhost{\gamma_{T}}{\authfull{#1}}}
\newcommand{\tracefrag}[1]{\ownGhost{\gamma_{T}}{\authfrag{#1}}}

The simplest way to do that is to directly expose the complete trace. In that
case, the state interpretation additionally holds a resource for the trace:

\[
  \X{state\_interp} \; (r,m,t) := \ldots \ast \tracefull{t}
\]

The user then works with assertions of the form $\tracefrag{t'}$. Such an
assertion is not duplicable, and grants full ownership over the trace. In
particular, it allows one to \emph{update} the trace by emitting events, i.e. by
performing I/O operations.

The wp-rules for \I{Load} and \I{Store} need to be updated consequently. For
instance, the rule for a \I{Load} reading the integer $x$ on a MMIO address $a$
now requires $\tracefrag{t}$ in the pre-condition (for some $t$), and provides
$\tracefrag{(\X{IORead}, a, x) :: t}$ in the post-condition.

The corresponding resource algebra is $\authm(\exm(\X{Trace}))$. A few relevant
rules are:

\[
  \begin{array}{l}
    \tracefull{t} \ast \tracefrag{t'} \wand \pure{t = t'} \\
    \tracefrag{t} \ast \tracefrag{t'} \wand \FALSE \\
    \tracefull{t} \ast \tracefrag{t} \vsW \tracefull{t'} \ast \tracefrag{t'} \\
  \end{array}
\]


\paragraph*{Remark:} This is very coarse-grained: either one has the full
ownership for performing I/O and reasoning about it, or one cannot know anything
about it.

A first extension could be to allow observing prefixes of the trace (since
events can only be appended to the trace). The observation that the trace
has a given prefix would be duplicable.

Another extension, that seems very useful for modular reasoning, would be to
allow splitting the trace along separate range of addresses. Concretely, a trace
containing events about the range of MMIO addresses $[a,c)$ could be split into
two traces, granting ownership over events on addresses $[a,b)$ and $[b,c)$
respectively. Note that recombining these two traces would only yield some
unspecified interleaving of the events from the two traces, and not necessarily
yield the original trace.

One application of this second extension could be the verification of an example
involving ``multiplexed'' I/O, where two separate parts of the code are granted
separate ownership over separate MMIO addresses. These two separate pieces of
code would be verified separately; then, in the end, one could prove that one
gets \emph{some} interleaving of all the events emitted by both components.


\section{Toplevel Theorem (\emph{à la} OCPL)}

Let us start with a brief recap of the toplevel ``Robust Safety'' theorem for
OCPL itself (and its key ingredients), then the extension (by Thomas) of OCPL to
include a \I{print} capability, and finally move to the capability machine
setting.

\subsection{OCPL}

The {\sc RobustSafety} theorem of OCPL is as follows:

\begin{mathpar}
  \mprset{vskip=0.3em}
  \inferrule
  {C \in \textit{AdvCtx} \\
    e \; \X{closed} \\
    \hoare{\TRUE}{e}{x.\; \textlog{lowval}\; x} \\
    (C[e]);(\emptyset, \textlog{OK}) \longrightarrow^* T';(h',g')
  }
  {g' = \textlog{OK}}
\end{mathpar}

For any closed expression $e$, if $e$ has been verified to only return low
values, then running $e$ wrapped in an adversarial context $C$ from an initial
state, then we can observe that every reachable state is good
($g' = \textlog{OK}$ means that no assertion has failed in $e$).

$C \in AdvCtx$ means that $C$ cannot contain assertions (otherwise one could
trivially contradict the theorem by taking
$C[\cdot] = \textlog{assert false}$), and cannot contain references to raw
memory locations (otherwise one could directly access $e$'s private state,
invalidating the local state encapsulation mechanisms).

\newcommand{\lift}[2]{\textlog{lift}\; #1 \; #2}
\newcommand{\liftP}[1]{\lift{\Psi}{#1}}

$\textlog{lowval}\; x$ intuitively means that $x$ cannot be used to directly
access private (or ``high'') locations. It is formally defined using a logical
relation ``$\liftP{v}$'', which more generally asserts that the value $v$ only
gives direct access to locations that satisfy the predicate $\Psi$.

\[
  \begin{array}{lcl}
    \liftP (\textlog{rec}\; f\; x.\; e) & \eqdef
    & \later \forall v.\; \hoare{\liftP{v}}{e[v/x,\textlog{rec}\;f\;x.\; e/f]}{y.\; \liftP y} \\
    %
    \liftP (v_{1}, v_{2}) & \eqdef & \later (\liftP v_{1}, \liftP v_{2}) \\
    %
    \color{BrickRed} \liftP \ell & \color{BrickRed} \eqdef & \color{BrickRed} \Psi \; \ell \\
    %
    \ldots
  \end{array}
\]

Then, $\textlog{lowval}\; x$ is defined as $\lift{\textlog{lowloc}}{x}$, where
$\textlog{lowloc}$ is a predicate characterizing the region of memory containing
``low locations'' (distinct from the other region containing ``high
locations'').

\subsection{OCPL with \I{print}}

\newcommand{\OutV}{\textlog{Out}}

In Thomas' extension of OCPL, a new value $\OutV$ is added, denoting an
``output object capability'', as well as a $\textlog{print}$ primitive, where
$\textlog{print}\; \OutV\; v$ effectively ``prints'' the value $v$, i.e. adds it
to the trace of printed values.

One then wants to be able to encapsulate the use of $\OutV$, for instance by
defining object capabilities that enforce some invariants on the values being
printed. In that setting, $\OutV$ is considered as a ``high'' value:

\FIXME{Thomas, I realize that this is less general that what you have (I
  think?), since in your case $\OutV$ \emph{can} be low, if the trace predicate
  is equivalent to True. But I don't know how to state that in a setting where
  the predicate is not directly embedded in the definitions...}

\[
  \begin{array}{lcl}
    \liftP \OutV & \eqdef & \FALSE
  \end{array}
\]

The theorem then becomes:

\begin{mathpar}
  \mprset{vskip=0.4em}
  \inferrule
  {C \in \textit{AdvCtx} \\
    e \; \X{closed} \\
    \knowInv{\iota}{\exists t.\; \tracefrag{t} \ast P(t)} \vdash \hoare{\TRUE}{e}{x.\; \textlog{lowval}\; x} \\
    (C[e]);(\emptyset, \textlog{OK}); \emptyset \longrightarrow^* T';(h',g'); t
  }
  {g' = \textlog{OK} \wedge P(t)}
\end{mathpar}

That is, if $e$ has been verified under the assumption that the predicate $P$
holds as a trace invariant, then executing $e$ in an adversarial context yields
a trace that does satisfy $P$.

$C \in \textit{AdvCtx}$ also has to be extended to forbid $C$ from containing
$\OutV$.

\subsection{Capability Machine with MMIO}

\newcommand{\VR}{\mathcal{V}}
\newcommand{\ER}{\mathcal{E}}
\newcommand{\RR}{\mathcal{R}}
\newcommand{\notMMIO}{\overline{\MMIO}}

Similarly to $\textlog{lift}$, we can generalize the existing logical relation
to thread a constraint on directly accessible memory locations. One would
parameterize the value, expression and register relations ($\VR$, $ER$ and $RR$)
with a predicate $\Psi$ on addresses.

Then, for any permission $p$ that includes either the R or W bit, $\VR$
is extended as follows:

\[
  \VR^{\Psi}(p, g, b, e, a) \eqdef \underbrace{\ldots}_{\text{as before}} \ast \; \pure{\forall a' \in [b,e).\; \Psi(a')}
\]

Then, $\ER^{\Psi}$ and $\RR^{\Psi}$ are simplify defined by threading the extra
$\Psi$ parameter through the existing definition.

Finally, for our relation to characterize ``low values'' that do not give direct
access to MMIO addresses, one would instantiate $\Psi$ with a predicate
$\notMMIO$ that excludes memory mapped addresses:

\[
  \notMMIO(a) \eqdef a \notin \MMIO
\]

Then, $\VR^{\notMMIO}$ is similar to the $\textlog{lowval}$ predicate of OCPL.
Intuitively, a value in the relation does not directly point to memory-mapped
locations, and also does not leak memory-mapped location through registers when
interacting with other code.

\paragraph*{Remark:} As stated above, the ``generalized'' value relation is in
fact not very useful for instantiation of $\Psi$ other than $\notMMIO$. Indeed,
even if $\Psi$ \emph{does} allow referring to addresses in the $\MMIO$ region,
the value relation does not grant any corresponding resources.
The fix would be to use the generalized trace resources mentioned previously,
and grant ownership for the part of the trace corresponding to addresses in
$[b,e) \cap \MMIO$.

Then, the toplevel theorem would be:

\begin{mathpar}
  \mprset{vskip=0.4em}
  \inferrule
  { \knowInv{\iota}{\exists t.\; \tracefrag{t} \ast P(t)} \vdash \VR^{\notMMIO}
    (E, g, b, e, a) \\
    (r[\X{PC} := (E, g, b, e, a)], m, \emptyset) \longrightarrow^*
    (r', m', t)
  }
  {P(t)}
\end{mathpar}

\FIXME{We need to add refine the assumptions of the theorem so that it has a
  chance of being true. One needs to add either a ``syntactic condition'' over
  thee state of the initial memory, or link with an initial boot-code and assume
  require the boot-code has a specification that entails the syntactic
  condition.}

\bibliographystyle{alpha}
\bibliography{biblio}

\appendix

\end{document}
